\documentclass[14pt, a4paper]{extarticle}
% Русская локализация
\usepackage[english,russian]{babel}


\usepackage{appendix}
\usepackage{alphabeta}

% Использование математических шрифтов
\usepackage{unicode-math}

% Шрифты
\usepackage{fontspec}
\usepackage{courier}
\defaultfontfeatures{Ligatures={TeX},Renderer=Basic}
\setmainfont[Ligatures={TeX}]{Times New Roman}
\setmonofont{Courier New}
\setmathfont{XITS Math}

% Расширенные ссылки
\usepackage{nameref}

% Оформление URL
\usepackage{xurl}
\usepackage{hyperref}
\hypersetup{
  colorlinks,
  citecolor=black,
  filecolor=black,
  linkcolor=black,
  urlcolor=black,
  breaklinks=true,
}
\urlstyle{same}

% Поддержка изображений
\usepackage{graphicx}
\graphicspath{{./images/}}
\DeclareGraphicsExtensions{.jpg,.png}
\usepackage{svg}

% Таблицы
\usepackage{tabularx}
\usepackage{tabulary}
\usepackage{ltablex}
\usepackage{multirow}
\usepackage{hhline} 
% Выравнивание по левому краю, с многострочностью
\newcolumntype{s}{>{\raggedright\arraybackslash}X}

% Поддержка листингов
\usepackage{listings}
\lstdefinestyle{gost}{
  basicstyle=\ttfamily\footnotesize,
  breakatwhitespace=false,
  breaklines=true,
  keepspaces=true,
  showspaces=false,          
  showstringspaces=false,
  frame=single
}
\lstset{style=gost}%

% Отступ первой строки первого абзаца
\usepackage{indentfirst}
\linespread{1.25}

% Размер полей в документе
\usepackage{geometry}
\geometry{left=3cm}
\geometry{right=1cm}
\geometry{top=2cm}
\geometry{bottom=2cm}

% Абзацный отступ
\setlength{\parindent}{1.25cm}

% Отступ для элементов в списке
\usepackage{enumitem}
\setlist{left=\parindent, labelsep=1cm, itemsep=0pt, topsep=0pt}

% Загрузка pdf-документов (нужно для титульных листов)
\usepackage[final]{pdfpages}
% Возможность поворота pdf файло
\usepackage{pdflscape}
\usepackage{everypage}

\newcommand{\Lpagenumber}{\ifdim\textwidth=\linewidth\else\bgroup%
    \dimendef\margin=0 %use \margin instead of \dimen0
    \ifodd\value{page}\margin=\oddsidemargin
    \else\margin=\evensidemargin%
    \fi
    \raisebox{\dimexpr-\topmargin-\headheight-\headsep-0.5\linewidth}[0pt][0pt]{%
      \rlap{\hspace{\dimexpr-\margin+\textheight+\footskip}%
        \llap{\rotatebox{90}{\thepage}}}}%
    \egroup\fi}
\AddEverypageHook{\Lpagenumber}%

\usepackage{float}
% Форматирование подписей
\usepackage{caption}

\usepackage{newfloat}
% \DeclareCaptionType{listing}

\DeclareCaptionLabelSeparator{emdash}{\;\textemdash\;}
\captionsetup[figure]{name={Рисунок}, labelsep=emdash, justification=centering, position=above, singlelinecheck=off, font={small, bf}, labelfont=bf, skip=6pt}
\captionsetup[table]{name={Таблица}, labelsep=emdash, justification=raggedright, position=top, singlelinecheck=off, font={small, it}, labelfont=it, skip=6pt, margin=0cm}
% \captionsetup[lstlisting]{labelsep=emdash, justification=raggedright, position=top, singlelinecheck=off, font={small, it}, labelfont=it, skip=6pt, margin=0cm}

% Нумеровать внутри заголовков первого уровня
\counterwithin{figure}{section}
\counterwithin{table}{section}
% \counterwithin{lstlisting}{section}
\AtBeginDocument{\counterwithin{lstlisting}{section}}

% Отключение переносов текста
\usepackage{ragged2e}
\justifying
\tolerance=500
\hyphenpenalty=10000
\emergencystretch=3em

% Форматирование заголовков
\usepackage{titlesec}
% Оформление заголовка первого уровня
% Полужирное начертание
% Кегль 18 пт
% С новой страницы
\titleformat{\section}[block]
{\newpage\bfseries\fontsize{18pt}{21.6pt}\selectfont}
{\thesection}
{1em}{}
% Оформление ненумерованных заголовков (Введение, Содержание, список источников, и.т.д.)
\titleformat{name=\section,numberless}[block]
{\centering\newpage\bfseries\fontsize{18pt}{21.6pt}\selectfont}
{}
{0em}{}{}
% Отступы у заголовков первого уровня
\titlespacing{\section}
{\parindent}% отступ слева (равен 1.25 см, как у отступа первой строки абзаца)
{0em}% интервал перед
{10mm}% интервал после
% Оформление заголовков второго уровня
\titleformat{\subsection}[block]
{\bfseries\fontsize{16pt}{19.2pt}\selectfont}
{\thesubsection}
{1em}{}
% Отступы у заголовков второго уровня
\titlespacing{\subsection}
{\parindent}% пробел слева
{15mm}% отступ перед
{10mm}% отступ после
% Оформление заголовков второго уровня
\titleformat{\subsubsection}[block]
{\bfseries\selectfont}
{\thesubsection}
{1em}{}
% Отступы у заголовков второго уровня
\titlespacing{\subsubsection}
{\parindent}% пробел слева
{15mm}% отступ перед
{10mm}% отступ после

% Оформление заголовков в содержании
\usepackage{titletoc}
\contentsmargin{0pt}
\renewcommand\contentspage{\thecontentspage}
\dottedcontents{section}[2.3em]{}{2.3em}{5pt}
\dottedcontents{subsection}[2.3em]{}{2.3em}{5pt}
% Оформление приложений
\usepackage{appendix}
\renewcommand\appendixpagename{ПРИЛОЖЕНИЯ}

% Подключение biblatex, с использованием стиля gost-numeric
\usepackage[
citestyle=gost-numeric,
style=gost-numeric, 
blockpunct=emdash,
backend=biber,
sorting=none
]{biblatex}
% Запрет разрыва url ссылок
\defcounter{biburlnumpenalty}{3000}
\defcounter{biburlucpenalty}{6000}
\defcounter{biburllcpenalty}{9000}
% Добавление полей для ссылок и даты обращения к ним
\DeclareFieldFormat{url}{Режим доступа: #1}
\DeclareFieldFormat{urldate}{(Дата обращения: #1)}
\renewcommand*{\entrysetpunct}{\par\nopunct\!\!}
% Использовать prac.bib как источник
\addbibresource{kurs.bib}
% Форматирование заголовка библиографии
\defbibheading{bibliography}[\bibname]{%
  \section*{\centering #1}%
  \markboth{#1}{#1}}


\usepackage{lipsum}
\begin{document}
\def\contentsname{СОДЕРЖАНИЕ}

% Загрузка титула
%\pagenumbering{gobble}
%\begin{titlepage}
%\includepdf{title}
%\includepdf[pages={1-2}]{zadanie}
%\end{titlepage}
%\pagenumbering{arabic}
%\setcounter{page}{4}
% Содержание
\tableofcontents

\section*{ВВЕДЕНИЕ}
\addcontentsline{toc}{section}{ВВЕДЕНИЕ}

Современные предприятия сталкиваются с необходимостью постоянного
совершенствования своих информационных систем для повышения
эффективности управления различными аспектами бизнеса. Одним из
ключевых направлений является оптимизация процессов взаимодействия с
партнерами и клиентами через внедрение современных технологий и
автоматизированных решений. В рамках данной курсовой работы
рассматривается проектирование ИТ-инфраструктуры предприятия, 
обеспечивающей функционирование модуля поддержки потребительского 
кредитования, участвующей в бизнес-процессе поддержки банковской 
информационной системы.

Актуальность темы обусловлена тем, что в условиях высокой конкуренции
и стремительного развития цифровых технологий компании вынуждены
адаптироваться к новым вызовам рынка. Это требует создания гибких и
масштабируемых инфраструктурных решений, способных поддерживать
сложные процессы взаимодействия между предприятием и его
партнёрами. Эффективная информационная система кредитования
должна обеспечивать оперативность обработки запросов, прозрачность
процедур, а также минимизировать риски ошибок и задержек.

Целью курсовой работы является разработка концепции проектирования
ИТ-инфраструктуры, способной обеспечить надежную работу информационной
системы кредитования, отвечая требованиям безопасности данных,
производительности и удобства использования. В работе будут
рассмотрены вопросы выбора программных и аппаратных средств,
интеграции различных компонентов инфраструктуры, а также обеспечения
соответствия разработанных решений современным стандартам и
нормативным актам.

Данная работа охватывает комплексное изучение объекта и предмета
исследования, проектирование и моделирование процессов с
использованием современных инструментов (UML и Archimate), а также
разработку календарно-ресурсного плана и анализ затрат на реализацию
проекта.

\section*{ГЛОССАРИЙ}
\addcontentsline{toc}{section}{ГЛОССАРИЙ}
\begin{raggedright}
	СХД --- Система хранения данных \\
	ИТ --- Информационные технологии \\
	ПАО --- Публичное акционерное общество \\ 
	CRM --- Customer Relationship Management \\
  	СУБД --- Система управления базами данных \\
  	ПО --- Программное обеспечение \\
\end{raggedright}

\section{Курсовая}
Паааехали

\end{document}